\documentclass{article}
\usepackage[pdftex]{graphicx} %for embedding images

\pagenumbering{roman}

\title{Building a Dynamic Website For Data Structures And Algorithms in Computer Science. }
\author{By\\ \centering
\begin{tabular}{|c|c|c|c|}
\hline
\textbf{No.}& \textbf{Student Name} & \textbf{RegNo} & \textbf{Std No} \\ \hline
\textit{1}&\textbf{AZAMUKE DENISH} & \textit{16/U/171}& \textit{216001004} \\ \hline
\textit{2}&\textbf{MUTAMBUZE PAUL}& \textit{16/U/7738/EVE}& \textit{216012181} \\ \hline
 \textit{3}&\textbf{OKOCHE GILBERT } & \textit{16/U/20063/EVE}  & \textit{216021710} \\ \hline
\textit{4}&\textbf{NSUBUGA FRANCIS} & \textit16/U/19985/EVE & \textit{216021708} \\ 
 \hline
\end{tabular}
\thanks{supervisor: Ernest Mwebaze}}
\date{%
    Makerere University\\%
    march 10 2018
}
\begin{document}

\begin{titlepage}
\maketitle
\end{titlepage}
\pagenumbering{arabic}

\section{Abstract}
\paragraph{Data structures are a conceptually demanding topic which confronts many Computer Science students early in their course. The topic has a strong conceptual basis and often proves difficult for many to grasp. A number of previous studies have examined that the use of interaction and visualization within the systems can motivate a student to engage in the learning process. This literature review investigates the effectiveness of these systems that were and are being used today for teaching and learning of data structures to novice Computer Science students.}
\paragraph{In this report a web application that features the visualization of commonly used data structures and their associated insertion and deletion operations is introduced.}
\section{Introduction}
\paragraph{Data Structures and Algorithms is a fundamental course in Computer Science. However, many students find it difficult because it requires abstract thinking. It would be very helpful if there was a visualization tool of data structures such as arrays, queues, stacks, trees and graphs for students to experiment with. The tool would allow students to see how an element is inserted into or deleted from different data structures, how a tree is traversed in different order (pre-order, in-order, post-order, level-order), etc. Moreover, this tool would provide a simple language, by which students can write their own algorithms so that the execution of the algorithm is animated. This project is intended to create such an exploration environment, in which students can learn through experimentation. This tool can be used as an effective supplement to the traditional lecture room education and textbooks for Data Structures and Algorithms courses. The web application presented in this document has the following functionality; Provides complete visualization for the widely used data structures such as array, stack, queue, tree, heap, graph, etc. Provides the animation of common operations associated with the data structures, such as inserting an element into and deleting an element from array, stack, and queue.Provides animation of simple user-defined algorithms.}




\end{document}